%% LyX 2.3.6 created this file.  For more info, see http://www.lyx.org/.
%% Do not edit unless you really know what you are doing.
\documentclass[twoside,english]{elsarticle}
\usepackage[T1]{fontenc}
\usepackage[latin9]{inputenc}
\pagestyle{headings}

\makeatletter
%%%%%%%%%%%%%%%%%%%%%%%%%%%%%% User specified LaTeX commands.
% specify here the journal
\journal{Example: Nuclear Physics B}

% use this if you need line numbers
%\usepackage{lineno}

\makeatother

\usepackage{babel}
\begin{document}

\begin{frontmatter}{}

\title{Accelerating QAOA}

\author[rvt]{Snir Bachar}

\ead{snir.bachar@campus.technion.ac.il}

\address[rvt]{Computer Science dept., Technion, Haifa}
\begin{abstract}
{\footnotesize{}The Quantum Approximate Optimization Algorithm (QAOA)
is a promising quantum algorithm for solving combinatorial optimization
problems. A the main aspect of QAOA's is the selection of its variational
parameters. Traditional methods for parameter optimization require
an efficient hybrid Quantum Computer, which is not an easy thing to
fetch. In this study, we introduce a novel approach to predict the
optimal parameters for QAOA circuits using a neural network model
that combines the spatial feature extraction capabilities of Convolutional
Neural Networks (CNNs), specifically the LeNet architecture, with
the sequence processing strengths of Long Short-Term Memory (LSTM)
layers. The LSTM integration allows our model to accommodate quantum
circuits of varying sizes, making it a versatile tool for diverse
quantum tasks. Our results indicate that by predicting near-optimal
parameters using the proposed model, we can significantly accelerate
the QAOA process while maintaining, or even enhancing, solution quality.
This research not only underscores the potential of classical-quantum
hybrid solutions in quantum computing but also offers a scalable approach
to optimize quantum circuits from a computer science perspective.}{\footnotesize\par}
\end{abstract}

\end{frontmatter}{}


\section{Introduction}

The Quantum Approximate Optimization Algorithm (QAOA) has emerged
as a pivotal quantum algorithm, especially in the noisy intermediate-scale
quantum (NISQ) era, characterized by limited qubits and heightened
error rates.

As a subset of variational quantum algorithms (VQAs), QAOA offers
a heuristic approach to tackle combinatorial optimization problems,
which have historically been computationally challenging. 

A distinguishing feature of VQAs, including QAOA, is their reliance
on shallow quantum circuits, rendering them relatively noise-resilient.

The efficacy of QAOA is intrinsically tied to the precise selection
of its variational parameters. 

While the optimization of these parameters is paramount, the traditional
methodologies often necessitate the deployment of hybrid Quantum Computers,
a resource that remains challenging to access widely. 

Each optimization iteration in QAOA can involve numerous circuit executions,
underscoring the need to minimize the number of such iterations.

In light of these challenges, this paper introduces a novel approach
leveraging classical machine learning techniques. 

By integrating the spatial feature extraction capabilities of Convolutional
Neural Networks (CNNs), exemplified by the LeNet architecture, with
the sequence processing prowess of Long Short-Term Memory (LSTM) layers,
we propose a model capable of predicting optimal QAOA parameters. 

This hybrid neural network model is designed to accommodate quantum
circuits of varying sizes, offering a versatile solution for a range
of quantum tasks.

Our primary objective is to diminish the number of QAOA iterations
without compromising on the algorithm's performance or incurring additional
quantum circuit executions. By utilizing a neural network to predict
initialization parameters tailored to each problem instance, we aim
to expedite the QAOA process, enhancing both its efficiency and solution
accuracy.

The contributions of this study are manifold:
\begin{enumerate}
\item We demonstrate the feasibility of predicting QAOA's variational parameters
using a hybrid CNN-LSTM model for problems like MaxCut, without necessitating
additional quantum computations. 
\item Empirical evaluations reveal that our proposed model surpasses existing
QAOA initialization techniques in terms of solution quality and convergence
rate. 
\item The approach potentially obviates the need for further quantum circuit
optimization, leading to significant computational savings. 
\item We observe that the benefits of our method amplify as the problem
size (e.g., the number of nodes in a graph) escalates.
\end{enumerate}
The remainder of this paper is structured as follows: Section 2 offers
a comprehensive background on QAOA, detailing its algorithmic structure
and its relevance in the context of the MaxCut problem. 

Section 3 delineates our proposed hybrid neural network model and
its training. 

In Section 4, we benchmark our approach against existing initialization
techniques, providing a comparative analysis of their performances. 

Finally, Section 5 delves into the implications of our method in the
broader landscape of the NISQ era.

\section{Quantum Approximate Optimization Algorithm}

The Quantum Approximate Optimization Algorithm (QAOA) has emerged
as a pivotal tool in the quantum computing domain, especially for
addressing combinatorial optimization problems. To understand its
significance and operation, it's essential to delve into its roots
in adiabatic quantum computation and its specific design for problems
like MaxCut.

\subsection{Adiabatic Quantum Computation}

Adiabatic Quantum Computation (AQC) is a quantum computing paradigm
that harnesses the adiabatic theorem of quantum mechanics. 

The core principle behind AQC is the slow evolution of a quantum system. 

Initially, the system is placed in the ground state of a simple Hamiltonian,
for which the ground state is known.

The Hamiltonian is then adiabatically evolved into a more complex
one, corresponding to the problem in question. 

If this evolution is sufficiently slow, the system remains in the
ground state, thus providing the solution to the problem.

The efficiency of AQC is determined by the gap between the ground
state and the first excited state of the Hamiltonian. 

A larger gap ensures a faster adiabatic evolution, leading to quicker
problem solutions. 

AQC has been explored for various optimization problems, laying the
groundwork for the development of algorithms like QAOA.

\subsection{The Quantum Approximate Optimization Algorithm }

The Quantum Approximate Optimization Algorithm (QAOA) can be viewed
as a quantum circuit-based approximation to AQC. 

Instead of a continuous adiabatic evolution, QAOA employs a discrete
set of unitary transformations, parameterized by angles, to approximate
the adiabatic pathway.

The QAOA circuit is characterized by its depth, denoted by $p$. 

For each layer, two sets of gates are applied: one corresponding to
the problem Hamiltonian (often denoted by $H_{P}$) and the other
to a mixing Hamiltonian (usually denoted by $H_{M}$). 

The angles parameterizing these gates, typically represented as $\beta$
and $\gamma$, are optimized classically to minimize the expectation
value of the problem Hamiltonian.

The strength of QAOA lies in its hybrid nature, combining quantum
circuits for generating states and classical routines for parameter
optimization. 

This synergy allows QAOA to be particularly effective on NISQ devices,
which have limited qubits and are susceptible to noise.

\subsection{Solving MaxCut with QAOA}

MaxCut is a classical combinatorial optimization problem where the
objective is to partition the nodes of a graph into two sets, maximizing
the number of edges that cross between the sets. 

Given its NP-hard nature, finding exact solutions for large graphs
is computationally challenging.

QAOA offers a quantum heuristic for the MaxCut problem. 

The problem Hamiltonian $H_{P}$ for MaxCut is constructed such that
its ground state corresponds to the optimal solution of the problem. 

The mixing Hamiltonian $H_{M}$ is typically chosen to be the transverse
field, promoting quantum fluctuations that help explore the solution
space.

By preparing a superposition of all possible cuts and evolving this
state under the influence of $H_{P}$ and $H_{M}$, QAOA navigates
the solution landscape. The variational parameters $\beta$ and $\gamma$
are then optimized to maximize the number of cut edges, using classical
optimization routines.

Once the optimal parameters are found, the QAOA circuit is executed
on a quantum processor, and measurements yield a cut. 

Repeated executions provide a distribution of cuts, from which the
best one can be selected.

\appendix

\section{Appendix name}

Appendix, only when needed.

\section*{\textemdash \textemdash \textemdash \textemdash \textemdash \textendash{}}

You can use either Bib\TeX :

\bibliographystyle{elsarticle-harv}
\addcontentsline{toc}{section}{\refname}\bibliography{xampl}


\section*{\textemdash \textemdash \textemdash \textemdash \textemdash \textemdash \textemdash{}}

\noindent Or plain bibliography:
\begin{thebibliography}{1}
\bibitem{key-1}Frank Mittelbach and Michel Goossens: \emph{The \LaTeX{}
Companion Second Edition.} Addison-Wesley, 2004.

\bibitem{key-2}Scott Pakin. The comprehensive \LaTeX{} symbol list,
2005.
\end{thebibliography}

\end{document}
